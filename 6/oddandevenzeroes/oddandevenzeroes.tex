\documentclass{article}

\title{Odd and Even Zeroes}
\author{Alexandre Charland}
\date{October 10, 2024}
\usepackage{amsmath}
\usepackage{pythonhighlight}
\begin{document}

\maketitle

\section{Explanation}
The numbers of zeros at the end of n! is the same question as how many factor of 10 does n! have? This is the same question has how many multiple of 2 and 5 does n! have? Since 2 is smaller than 5, the amounts of 2 will always be higher than the amounts of 5. The question becomes, is there an even number of multiple of 5 in n!.\\
An easy solution to the problem would be
\begin{python}
a=[0,0,0,0,1]
n=int(input())
while(n>len(a)):
    a=a*5
    a[-1]+=1
m=1
t=1
for i in range(n):
    t+=a[i]
    m+=(t%2)
print(m)
\end{python}
However the array a will get very big for bigs numbers. So it isn't a valid solution, because of a Memory Limit Exceeded error.\\
I use this pseudocode to find motif and came up with this ugly (or beautiful) formula to solve the problem. To use it the number must be in base 5. The formula can be calculated in O(log(N)).\
\[ n=a_{k}\cdot 5^{k}+a_{k-1}\cdot 5^{k-1}+\ldots+a_{1}\cdot 5 +a_{0} \]
Where $0 \le a_{i} < 5, \forall i \in [1,k]$ and $a_{k} \neq 0$\\
Let's also define $n_i = n\% 5^{i+1}, \forall i \in [1,k]$

\[ f(n_{i}) = \left\{
     \begin{array}{@{}l@{\thinspace}l}
       f_{a_{i}}(i)+1+n_{i-1}-f(n_{i-1})  &:\text{if } i \text{ and } a_{i} \text{ are odds}\\
       f_{a_{i}}(i)+f(n_{i-1}) &: \text{otherwise}\\
     \end{array}
   \right. \]
\[ f_{0}(i)=0 \]
\[ f_{1}(i)=\left\{
     \begin{array}{@{}l@{\thinspace}l}
       1 &:\text{if } i=0\\
       f_{1}(i-1)\cdot 5 &: \text{if } i \text{ is odd}\\
       f_{1}(i-1)-2\cdot 5^{\frac{i-2}{2}} &: \text{if } i \text{ is even}\\
     \end{array}
   \right. \]
\[ f_{2}(i)=\left\{
    \begin{array}{@{}l@{\thinspace}l}
      5^{i} &: \text{if } i \text{ is odd}\\
      5^{i}+5^{\frac{i}{2}} &: \text{if } i \text{ is even}\\
    \end{array}
  \right. \]
\[ f_{3}(i)=\left\{
    \begin{array}{@{}l@{\thinspace}l}
      (g(i)-1)\cdot 5^{\frac{i}{2}} &: \text{if } i \text{ is odd}\\
      (g(i)-1)\cdot 5^{\frac{i+1}{2}}\cdot 2 &: \text{if } i \text{ is even}\\
    \end{array}
  \right. \]
  \[ g(i)=\left\{
    \begin{array}{@{}l@{\thinspace}l}
      4 &: \text{if } i=0\\
      \frac{g(i-1)}{2} &: \text{if } i \text{ is odd}\\
      g(i-2)\cdot 5 - 10 &: \text{if } i \text{ is even}\\
    \end{array}
  \right. \]
\[ f_{4}(i) = 2\cdot f_{2}(i) \]
\end{document}